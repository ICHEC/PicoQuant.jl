\title{QuantEx: Interim Report 1}
\author{
        QuantEx Team at ICHEC
}
\date{\today}

\documentclass[12pt]{article}

\begin{document}
\maketitle

\begin{abstract}
Working document for first QuantEx interim report in which we keep track of
background reading, exploration work and planning process to arrive at initial
prototype design.
\end{abstract}

\section{Introduction}
in this report we track the progress of the initial stages of the QuantEx project. In section \ref{evaluations} report on evaluations of the currently existing tools and methods for simulating quantum circuits using tensor networks. For each, we assess their potential to be integrated and/or modified to form part of the QuantEx toolset and to scale to emerging exa-scale compute platforms. In section \ref{design} we outline the interfaces and architecture of each layer that will be used for the initial implementation.

\section{Evaluations}\label{evaluations}
Here we go through each of the relevant packages and assess their applicability for use in the QuantEx project.

\subsection{qflex}
The qflex simulator received a lot of attention for its use in the google quantum supremacy experiments \cite{Villalonga2019} and managed to reach sustained performance of 281 Pfops/s on the Summit supercomputer at Oakridge National Laboratory.... 

\section{Design of initial implementation}\label{design}

\section{Conclusions}\label{conclusions}

\bibliographystyle{habbrv}
\bibliography{library}

\end{document}
This is never printed